\section{Cantorräume}
Wir werden im Folgenden zelluläre Automaten als dynamische Systeme auf speziellen kompakten metrisierbaren Räumen betrachten, den Cantorräumen.
\begin{erinnerung}
  Eine Menge $X$ heißt \markdef{topologischer Raum mit Topologie} $\cT \subseteq \cP(x)$, falls $\emptyset, X \in \cT$ und jede beliebige Vereinigung und jeder beliebige Schnitt von Mengen aus $\cT$ wieder in $\cT$ liegt. Die Elemente in $\cT$ heißen offene Mengen. Eine Teilmenge von $X$ heißt abgeschlossen, wenn ihr Komplement offen ist, Jeder metrische Raum ist auch topologischer Raum. Ein topologischer Raum heißt zusammenhängend, wenn seine einzigen offenen abgeschlossenen Teilmengen  die leere Menge und der Raum selbst ist. Die \markdef{Produkttopologie} isdt die gröbste Topologie (mit den wenigsten offenen Mengen) bezüglich welcher alle Projektionen $p_i: \prod A_k \to A_i$ stetig sind. 
\end{erinnerung}
\begin{definition} Mitteldrittel-Cantormenge

Wir betrachten eine geschachtelte Folge von Teilmengen von $[0, 1]$:
\begin{align*}
  C_0 &= [0,1]\\
  C_ {n+1}& = \frac 13 C_n \cup \left( \frac 2 3 + \frac 1 3 C_n\right), \quad n \in \N.
\end{align*}
 Jede dieser Mengen $C_n$ ist die disjunkte Vereinigung von abgeschlossenen Intervallen der Länge $\frac 1 {3^n}$ und $C_{n+1}$ entsteht aus $C_n$, indem aus jeder dieser Intervalle das mittlere Drittel entfernt wird. Die \markdef{Mitteldrittel-Cantormenge}
 \begin{align*}
   C = \bigcap_{n \in \N_0} C_n
 \end{align*}
ist der Durchschnitt dieser Mengen.
\end{definition}
$C$ enthält die Endpunkte aller Intervalle in $C_n$, aber darüber hinaus noch viele weitere Punkte. 
\begin{uebung}
  Es gilt
  \begin{align*}
    C = \frac 1 3C \cup \left( \frac 2 3 + \frac 1 3 C\right)
  \end{align*}
und 
\begin{align*}
  C \subseteq \bigcup_{i = 0}^{3^n-1}\left( \frac 1 {3^n} + \frac 1{3^n}C\right).
\end{align*}
\end{uebung}
\begin{bemerkung}
  Die Elemente von $C$ sind genau die Zahlen im Intervall $[0, 1]$, in deren Ternär-Darstellung, also zur Basis $3$, nur die Ziffern $0$ und $2$ vorkommen. 

Erinnerung: Zu jedem $r \in [0, 1]$ gibt es eine Folge $\left(a_k\right)_{k \in \N} \subset \set {0, 1, 2}$, mit 
\begin{align*}
  r = \sum_{k = 1}^ \infty \frac {a_k}{3^k}.
\end{align*}
Wenn man sich auf solche Folgen beschränkt, die nicht schließlich konstant $2$ sind, ist diese Darstellung eindeutig für $r \in [0,1)$. Zum Beispiel ist die Ternärdarstellung von 
\begin{align*}
  \frac 5 8 = 0,\overline{12} = \frac 1 3 \sum_{k = 0}^\infty \frac 1 {9^k} + \sum_{k = 0}^\infty  \frac 2 {9^k}.
\end{align*}
\end{bemerkung}
Wir halten nun einige Eigenschaften von $C$ fest. 
\begin{lemma} Eigenschaften der Cantormenge
  
Die Menge $C$ ist 
\renewcommand{\labelenumi}{(\alph{enumi})}
\begin{enumerate}
\item kompakt, 
\item perfekt, das heißt jeder Punkt ist Häufungspunkt, 
\item Lebesgue-messbar und hat Lebesguemaß $0$, 
\item total unzusammenhängend, das heißt, jede Teilmenge mit mindestens zwei Punkten ist unzusammenhängend, 
\item null-dimensional, das heißt jede offene Menge ist als Vereinigung von abgeschlossenen offenen Mengen darstellbar.  
\end{enumerate}
\end{lemma}
\begin{beweis}
\renewcommand{\labelenumi}{(\alph{enumi})}
  \begin{enumerate}
  \item $C$ ist beschränkt und als Durchschnitt abgeschlossener Mengen abgeschlossen, also kompakt. 
  \item Sei $\eps >0$ und $n \geq 0$ so, dass $\frac 1 {3^n} \leq \eps$. Dann sei $x \in C \subseteq C_n$ und $I$ das maximale Teilintervall von $C_n$, in dem $x$ liegt. Dann ist aber 
    \begin{align*}
      \max I - \min I = \frac 1 {3^n}.
    \end{align*}
    Damit haben aber beide Endpunkte von $I$ zu $x$ einen Abstand, der kleiner  oder gleich ist als $ \frac 1 {3^n}$. Es gibt also einen Punkt unglich $x$ in $C \cap B_\eps(x)$ für jedes $\eps >0$. $x$ ist also Häufungspunkt.
  \item $C$ ist als abzählbarer Durchschnitt Lebesgue-messbarer Mengen Lebesgue-messbar und aus 
    \begin{align*}
      \mu(C_n) = \left(\frac 2 3\right)^n
    \end{align*}
    folgt $\mu(C) = 0$.
  \item Sei $X$ eine echte Teilmenge von $C$, die mindestens zwei verschiedene Punkte $x$ und $y$ enthält, ohne Einschränkung $x < y$. Dann enthält $[x, y]$ aber einen Punkt $z$, der nicht in $X$ liegt, sonst wäre
    \begin{align*}
      \mu(C)\leq \mu([x, y]) = y -x
    \end{align*}
    im Widerspruch zu $\mu(C) = 0$. Damit sind $[0, z) \cap X$, $(z, 1] \cap X$ zwei disjunkte, abgeschlossene Mengen, deren Vereinigung $X$ ergibt (bezüglich der Relativtopologie in $X$!). 
  \item Wir zeigen zunächst, dass sich für jedes $\eps > 0$ eine Überdeckung von $C$ mit abgeschlossenen offenen Mengen mit Durchmesser kleiner $\eps$ finden lässt. Nach Aufgabe 2.3 gilt 
    \begin{align*}
      C \subset \bigcup_{i = 0}^n\left( \left( \frac i {3^n} + \frac 1 {3^n C}\right) \cap C\right).
    \end{align*}
Da das Bild einer offenen abgeschlossenen Menge unter einem Homöomorphismus wieder offen und abgeschlossen ist, stellen die verschobenen skalierten Bilder von $C$ uf der rechten Seite der Inklusion eine solche Überdeckung mit offenen abgeschlossenen Mengen mit Durchmesser kleiner oder gleich $\frac 1 {3^n}$ dar. Sei nun $A \subseteq C$ offen. Dann gibt es für jedes $x \in A$ ein $\eps > 0$ mit $B_\eps(x) \subseteq A$ und eine offene abgeschlossene Menge $D_x$ mit Durchmesser kleiner $\eps$ und $x \in D_x$. Damit gilt für alle $y \in D_x$, dass $\norm{x - y} < \eps$, also $D_x \subseteq B_\eps(x) \subseteq A$. Somit ist $A= \bigcup_{x \in A} D_x$.
\end{enumerate}
\end{beweis}
\begin{uebung} Äquivalenz von Null-Dimensionalität und Unzusammenhang

Sei $X$ ein kompakter metrischer Raum. Dann ist $X$ genau dann unzusammenhängend, wenn $X$ null-dimensional ist. (Definitionen reichen, oder Literatur)
\end{uebung}
\begin{definition}
Einen nichtleeren kompakten null-dimensionalen, perfekten metrischen Raum nennen wir \markdef{Cantorraum}.
\end{definition}
\begin{uebung} Symbolräume sind Cantorräume

Sei $(A_k)_{k \in M}$ mit $M = \N$ oder $M = \Z$ eine Folge von endlichen Mengen mit jeweils mehr als einem Element. Dann nennen wir die Menge $\prod_{k \in M} A_k$ zusammen mit der Metrik $d$ definiert durch 
\begin{align*}
  d\left((a_n), (b_n)\right) = 2^{- \min\set{\norm l: l \in M, \; a_l \neq b_l}}
\end{align*}
 einen \markdef{Symbolraum}. Zeigen Sie, dass jeder Symbolraum ein Cantorraum ist. 
\end{uebung}
\begin{bemerkung}
  Für $(a_n), (b_n)$ aus Aufgabe 2.8 gilt die Äquivalenz
  \begin{align*}
    a_l = b_l \text{ für } \norm l = 0, \dots, n &\Leftrightarrow d((a_n), (b_n)) < \frac 1 {2^n}\\
&\Leftrightarrow d((a_n), (b_n)) \leq \frac 1 {2^{n+1}}
  \end{align*}
\end{bemerkung}