\section{Cantorräume}
Wir werden im Folgenden zelluläre Automaten als dynamische Systeme auf speziellen kompakten metrisierbaren Räumen betrachten, den Cantorräumen.
\begin{erinnerung}
  Eine Menge $X$ heißt \markdef{topologischer Raum mit Topologie} $\cT \subseteq \cP(x)$, falls $\emptyset, X \in \cT$ und jede beliebige Vereinigung und jeder beliebige Schnitt von Mengen aus $\cT$ wieder in $\cT$ liegt. Die Elemente in $\cT$ heißen offene Mengen. Eine Teilmenge von $X$ heißt abgeschlossen, wenn ihr Komplement offen ist, Jeder metrische Raum ist auch topologischer Raum. Ein topologischer Raum heißt zusammenhängend, wenn seine einzigen offenen abgeschlossenen Teilmengen  die leere Menge und der Raum selbst ist. Die \markdef{Produkttopologie} isdt die gröbste Topologie (mit den wenigsten offenen Mengen) bezüglich welcher alle Projektionen $p_i: \prod A_k \to A_i$ stetig sind. 
\end{erinnerung}
\begin{definition} Mitteldrittel-Cantormenge

Wir betrachten eine geschachtelte Folge von Teilmengen von $[0, 1]$:
\begin{align*}
  C_0 &= [0,1]\\
  C_ {n+1}& = \frac 13 C_n \cup \left( \frac 2 3 + \frac 1 3 C_n\right), \quad n \in \N.
\end{align*}
 Jede dieser Mengen $C_n$ ist die disjunkte Vereinigung von abgeschlossenen Intervallen der Länge $\frac 1 {3^n}$ und $C_{n+1}$ entsteht aus $C_n$, indem aus jeder dieser Intervalle das mittlere Drittel entfernt wird. Die \markdef{Mitteldrittel-Cantormenge}
 \begin{align*}
   C = \bigcap_{n \in \N_0} C_n
 \end{align*}
ist der Durchschnitt dieser Mengen.
\end{definition}
$C$ enthält die Endpunkte aller Intervalle in $C_n$, aber darüber hinaus noch viele weitere Punkte. 
\begin{uebung}
  Es gilt
  \begin{align*}
    C = \frac 1 3C \cup \left( \frac 2 3 + \frac 1 3 C\right)
  \end{align*}
und 
\begin{align*}
  C \subseteq \bigcup_{i = 0}^{3^n-1}\left( \frac 1 {3^n} + \frac 1{3^n}C\right).
\end{align*}
\end{uebung}
\begin{bemerkung}
  Die Elemente von $C$ sind genau die Zahlen im Intervall $[0, 1]$, in deren Ternär-Darstellung, also zur Basis $3$, nur die Ziffern $0$ und $2$ vorkommen. 

Erinnerung: Zu jedem $r \in [0, 1]$ gibt es eine Folge $\left(a_k\right)_{k \in \N} \subset \set {0, 1, 2}$, mit 
\begin{align*}
  r = \sum_{k = 1}^ \infty \frac {a_k}{3^k}.
\end{align*}
Wenn man sich auf solche Folgen beschränkt, die nicht schließlich konstant $2$ sind, ist diese Darstellung eindeutig für $r \in [0,1)$. Zum Beispiel ist die Ternärdarstellung von 
\begin{align*}
  \frac 5 8 = 0,\overline{12} = \frac 1 3 \sum_{k = 0}^\infty \frac 1 {9^k} + \sum_{k = 0}^\infty  \frac 2 {9^k}.
\end{align*}
\end{bemerkung}
Wir halten nun einige Eigenschaften von $C$ fest. 
\begin{lemma} Eigenschaften der Cantormenge
  
Die Menge $C$ ist 
\renewcommand{\labelenumi}{(\alph{enumi})}
\begin{enumerate}
\item kompakt, 
\item perfekt, das heißt jeder Punkt ist Häufungspunkt, 
\item Lebesgue-messbar und hat Lebesguemaß $0$, 
\item total unzusammenhängend, das heißt, jede Teilmenge mit mindestens zwei Punkten ist unzusammenhängend, 
\item null-dimensional, das heißt jede offene Menge ist als Vereinigung von abgeschlossenen offenen Mengen darstellbar.  
\end{enumerate}
\end{lemma}
\begin{beweis}
\renewcommand{\labelenumi}{(\alph{enumi})}
  \begin{enumerate}
  \item $C$ ist beschränkt und als Durchschnitt abgeschlossener Mengen abgeschlossen, also kompakt. 
  \item Sei $\eps >0$ und $n \geq 0$ so, dass $\frac 1 {3^n} \leq \eps$. Dann sei $x \in C \subseteq C_n$ und $I$ das maximale Teilintervall von $C_n$, in dem $x$ liegt. Dann ist aber 
    \begin{align*}
      \max I - \min I = \frac 1 {3^n}.
    \end{align*}
    Damit haben aber beide Endpunkte von $I$ zu $x$ einen Abstand, der kleiner  oder gleich ist als $ \frac 1 {3^n}$. Es gibt also einen Punkt unglich $x$ in $C \cap B_\eps(x)$ für jedes $\eps >0$. $x$ ist also Häufungspunkt.
  \item $C$ ist als abzählbarer Durchschnitt Lebesgue-messbarer Mengen Lebesgue-messbar und aus 
    \begin{align*}
      \mu(C_n) = \left(\frac 2 3\right)^n
    \end{align*}
    folgt $\mu(C) = 0$.
  \item Sei $X$ eine echte Teilmenge von $C$, die mindestens zwei verschiedene Punkte $x$ und $y$ enthält, ohne Einschränkung $x < y$. Dann enthält $[x, y]$ aber einen Punkt $z$, der nicht in $X$ liegt, sonst wäre
    \begin{align*}
      \mu(C)\leq \mu([x, y]) = y -x
    \end{align*}
    im Widerspruch zu $\mu(C) = 0$. Damit sind $[0, z) \cap X$, $(z, 1] \cap X$ zwei disjunkte, abgeschlossene Mengen, deren Vereinigung $X$ ergibt (bezüglich der Relativtopologie in $X$!). 
  \item Wir zeigen zunächst, dass sich für jedes $\eps > 0$ eine Überdeckung von $C$ mit abgeschlossenen offenen Mengen mit Durchmesser kleiner $\eps$ finden lässt. Nach Aufgabe 2.3 gilt 
    \begin{align*}
      C \subset \bigcup_{i = 0}^n\left( \left( \frac i {3^n} + \frac 1 {3^n C}\right) \cap C\right).
    \end{align*}
Da das Bild einer offenen abgeschlossenen Menge unter einem Homöomorphismus wieder offen und abgeschlossen ist, stellen die verschobenen skalierten Bilder von $C$ uf der rechten Seite der Inklusion eine solche Überdeckung mit offenen abgeschlossenen Mengen mit Durchmesser kleiner oder gleich $\frac 1 {3^n}$ dar. Sei nun $A \subseteq C$ offen. Dann gibt es für jedes $x \in A$ ein $\eps > 0$ mit $B_\eps(x) \subseteq A$ und eine offene abgeschlossene Menge $D_x$ mit Durchmesser kleiner $\eps$ und $x \in D_x$. Damit gilt für alle $y \in D_x$, dass $\norm{x - y} < \eps$, also $D_x \subseteq B_\eps(x) \subseteq A$. Somit ist $A= \bigcup_{x \in A} D_x$.
\end{enumerate}
\end{beweis}
\begin{uebung} Äquivalenz von Null-Dimensionalität und Unzusammenhang

Sei $X$ ein kompakter metrischer Raum. Dann ist $X$ genau dann unzusammenhängend, wenn $X$ null-dimensional ist. (Definitionen reichen, oder Literatur)
\end{uebung}
\begin{definition}
Einen nichtleeren kompakten null-dimensionalen, perfekten metrischen Raum nennen wir \markdef{Cantorraum}.
\end{definition}
\begin{uebung} Symbolräume sind Cantorräume

Sei $(A_k)_{k \in M}$ mit $M = \N$ oder $M = \Z$ eine Folge von endlichen Mengen mit jeweils mehr als einem Element. Dann nennen wir die Menge $\prod_{k \in M} A_k$ zusammen mit der Metrik $d$ definiert durch 
\begin{align*}
  d\left((a_n), (b_n)\right) = 2^{- \min\set{\norm l: l \in M, \; a_l \neq b_l}}
\end{align*}
 einen \markdef{Symbolraum}. Zeigen Sie, dass jeder Symbolraum ein Cantorraum ist. 
\end{uebung}
\begin{bemerkung}
  Für $(a_n), (b_n)$ aus Aufgabe 2.8 gilt die Äquivalenz
  \begin{align*}
    a_l = b_l \text{ für } \norm l = 0, \dots, n &\Leftrightarrow d((a_n), (b_n)) < \frac 1 {2^n}\\
&\Leftrightarrow d((a_n), (b_n)) \leq \frac 1 {2^{n+1}}
  \end{align*}
\end{bemerkung}
%\datum{18. November 2014}

\begin{bemerkung}
  Den gleichen topologischen Raum wie in Aufgabe 2.8 erhält man, indem man das Produkt der diskreten topologischen Räume $(A_k, \cP(A_k))$ mit der Produkttopologie betrachtet.  
\end{bemerkung}
\begin{satz} EIndeutigkeit des Cantorraumes
 
  Jeder Cantorraum ist homöomorph zu $\set {0, 1}^\N$.
\end{satz}
Wir beweisen zunächst ein Lemma:
\begin{lemma} Abgeschlossene offene Mengen in Cantorräumen

Sei $X$ ein Cantorraum und sei $\cB$ die Menge der zugleich offenen und abgeschlossenen Teilmengen von $X$. 
\renewcommand{\labelenumi}{(\alph{enumi})}
\begin{enumerate}
\item Für jedes $x \in X$ und $\eps >0$ git es $B \in \cB$ mit $x \in B$ und $\diam(B)< \eps$.
\item Sei $\emptyset \neq Y \subseteq X$ abgeschlossen und offen. Dann ist $Y$ wieder ein Cantorraum und es gibt zwei nichtleere abgeschlossene offene Teilmengen, deren Vereinigung $Y$ ist.
\item Für jedes $\eps > 0$ gibt es ein $N \in \N$, sodass für alle $n > N$ möglich ist, dass $X$ als Vereinigung von $n$ nichtleeren abgeschlossenen offenen Mengen mit Durchmesser kleiner $\eps$ darzustellen. 
\end{enumerate}
\end{lemma}
\begin{beweis}
  \enu{\alph}
  \begin{enumerate}
  \item Die Kugel $B_{\frac \eps 2}(x)$ lässt sich als Vereinigung von offenen abgeschlossenen Mengen darstellen, also gibt es $B \in \cB$ mit $x \in B$ und $B \subseteq B_{\frac \eps 2}(x)$ und damit $\diam(B)< \eps$.
  \item Kompaktheit, Perfektheit und Null-Dimensionalität von $Y$ lassen sich leicht prüfen. Da $Y$ als perfekter metrischer Raum mindestens zwei Punkte enthält, gibt es $x, y \in Y$, $x \neq y$ und nach (a) gibt es $B \in \cB$ mit $\diam(B)< d(x, y)$ und $x \in B$, also ist $B \cap Y  \subsetneq Y$ abgeschlossen und offen und bildet mit $Y \setminus B$ eine Zerlegung von $Y$ in zwei offene abgeschlossene Mengen.
\item Nach (a) können wir $X$ mit Mengen aus $\cB$ mit Durchmesser kleiner $\eps$ überdecken, und da $X$ kompakt ist, gibt es eine endliche Teilüberdeckung. Inden man paarweise Durchschnitte bildet, erhält man eine Zerlegung von $X$ in $N \in \N$ offene abgeschlossene Mengen mit Durchmesser $\eps$. Indem man eine der Mengen mit (b) weiter zerlegt, erhält man schrittweise auch Zerlegungen in $n > N$ Mengen.
\end{enumerate}
\end{beweis}
\begin{beweis} von Satz 2.11: Wir zerlegen mithilfe von Lemma 2.12 (c) den Cantorraum $X$ in $2^{n_1}$ nichtleere abgeschlossene offene Teilmengen $V_1, \dots V_{2^{n_1}}$ mit Durchmesser gleich $\frac 1 {2^1}$. Diese zerlegen wir wiederum jeweils in $2^{n_2}$ nichtleere abgeschlossene offene Teilmengen 
  \begin{align*}
    &V_{1, 1}, \dots, V_{1, 2^{n_2}} \subseteq V_1\\
    &V_{2, 1}, \dots, V_{2, 2^{n_2}} \subseteq V_2\\
    &\qquad\dots\\
    &V_{2^{n_1}, 1}, \dots, V_{2^{n_1}, 2^{n_2}} \subseteq V_2^{n_1}
  \end{align*}
mit Durchmesser kleiner gleich $\frac 1 {2^2}$. Rekursiv fahren wir nun fort und erhalten Familien von nichtleeren abgeschlossenen Teilmengen 
\begin{align*}
  &(V_i)_{i \in \set{1, \dots , 2^{n_1}}}\\
  &(V_{i,j})_{i \in \set{1, \dots , 2^{n_1}}, j \in \set{1, \dots, 2^{n_2}}}\\
  &(V_{i,j, k})_{i \in \set{1, \dots , 2^{n_1}}, j \in \set{1, \dots, 2^{n_2}}, k \in \set{1, \dots, 2^{n_3}}}, \dots
\end{align*}
für die jeweils gilt
\begin{align*}
  &V_{a_1, \dots, a_l, a_{l+1}} \subseteq V_{a_1, \dots, a_l}\\
  &\diam(V_{a_1, \dots, a_l}) < \frac 1 {2^l}
\end{align*}
für $l \in \N$. Sei 
\begin{align*}
  (y_1, y_2, \dots) \in Y \coloneqq \Pi_{l \in \N} \set{1, \dots, 2^{n_l}}.
\end{align*} 
Dann ist 
\begin{align*}
  V_{y_1} \cap   V_{y_1, y_2} \cap   V_{y_1, y_2, y_3}  \cap \dots
\end{align*}
nichtleer als Schachtelung abgeschlossener Mengen in einem kompakten Raum und enthält genau einen Punkt, da der Durchmesser der Mengen gegen $0$ geht. Also können wir eine Abbildung $f: Y \to X$ definieren, wobei $f\left((y_i)_{i \in \N}\right)$ das eindeutig bestimmte Element in 
\begin{align*}
  \bigcap_{l \in \N} V_{y_1, \dots, ,y_l} 
\end{align*}
ist. Für $y, z \in Y$ mit $y \neq z$ gibt es ein $l \in \N$ mit $y_l \neq z_l$. Dann ist aber 
\begin{align*}
  V_{y_1, \dots y_l} \cap V_{z_1, \dots z_l} \neq \emptyset,
\end{align*}
also auch 
\begin{align*}
  f(y) \neq f(z).
\end{align*}
Damit ist $f$ injektiv. $f$ ist surjektiv, da für $x \in X$ und $l \in \N$ genau ein Tupel $(y_1, \dots, y_l)$ existiert mit $x \in V_{y_1, \dots, y_l}$. Damit gilt $x \in \bigcap_{l \in \N} V_{y_1, \dots y_l}$. 
Seien $y, z \in Y$ mit $d(y, z) \leq \frac 1 {2^n}$. Dann gilt 
\begin{align*}
 y_1 = z_1,\; \dots,\; y_n = z_n, 
\end{align*}
also ist $f(y) \in V_{y_1, \dots, y_n}$ und $f(z) \in V_{y_1, \dots y_n}$. Da $\diam(V_{y_1, \dots, y_n})< \frac 1 {2^n}$, gilt auch 
\begin{align*}
  d(f(y), f(z))< \frac 1 {2^n}.
\end{align*}g
Damit ist $f$ stetig und da $X$ und $Y$ kompakt sind, ist $f$ ein Homöomorphismus. Schließlich ist $\set{0,1}^n$ homöomorph zu $\set{1, \dots, 2^n}$ für $n \in \N$, also ist $Y$ homöomorph zu $\set{0,1}^\N$.
\end{beweis}
Damit ist insbesondere die Mitteldrittel-Cantormenge homöomorph zu $\set{0,1}^\N$. Hier kann man den Homöomorphismus sogar direkt angeben, nämlich 
\begin{align*}
  f:& \set {0,1}^\N \to C,\\
  &a_n \mapsto \sum_{n = 1}^\infty \frac {2a_n}{3^{n-1}}.
\end{align*}

%\datum{25. November 2014}

\begin{satz}
  Jeder kompakte metrische Raum $X$ ist Bild einer stetigen Abbildung von einem Cantorraum nach $X$.
\end{satz}
\begin{satz} Jedes (kompakte!) dynamische System $(X, \phi)$ ist Faktor eines dynamischen Systems auf einem Cantorraum.
\end{satz}
\begin{definition} Eine stetige Abbildung $f: X \to Y$ hat die \markdef{Erweiterungseigenschaft}, wenn es für jedes stetige $g:X \to Y$ ein $\phi: X \to X$ gibt, sodass $g = f \circ \phi$.
  \begin{align*}
    \left(  \forall g \in C(X, Y) \; \exists \phi: X \to X: g = f \circ \phi\right)
  \end{align*}
\end{definition}
\begin{definition} Sei $X$ ein metrischer Raum und $\cV$ eine offene Überdeckung von $X$. Dann heißt $\lambda >0$ \markdef{Lebesguezahl} von $\cV$, wenn es für jedes $x \in X$ ein $V \in \cV$ gibt mit $B_\l(x) \subseteq V$.
\end{definition}
\begin{uebung} Zeigen Sie, dass für jede offene Überdeckung eines kompakten metrischen Raumes eine Lebesguezahl existiert.
\end{uebung}
\begin{satz}(Weihrauch)

  Für jeden kompakten metrischen Raum $Y$ gibt es eine stetige surjektive Abbildung $f: \set{0,1}^\N \to Y$ mit der Erweiterungseigenschaft.
\end{satz}
\begin{beweis}
  Analog zum Beweis von Satz 2.11 konstruieren wir eine Familie von offenen Überdeckungen
  \begin{align*}
    (V_i)_{i \in \set{1, \dots, n_1}}, (V_{i, j})_{i \in \set{1, \dots, n_1}, j \in \set{1, \dots, n_2}}, \dots
  \end{align*}
  so, dass für alle $k \in \N$ und $(a_1, \dots, a_k) \in \set{1, \dots, n_1}\times \dots \times \set{1, \dots, n_k}$ gilt
  \begin{align*}
    V_{a_1, \dots, a_k} = \bigcup_{i \in \set{1, \dots, n_{k+1}}} V_{a_1, \dots, a_k,i}
  \end{align*}
  und $\diam(V_{a_1, \dots, a_k})<2^{-k}$. Wir definieren eine Folge von Mengen
  \begin{align*}
    A_k \coloneqq \prod_{l = 1}^k \set{1, \dots, n_l}
  \end{align*}
  und den symbolischen Raum
  \begin{align*}
    X \coloneqq \prod_{l = 1}^\infty\set{1, \dots, n_l}
  \end{align*}
  ausgestattet mit der Metrik aus Aufgabe 2.8.
  Dann ist $X$ homöomorph zu $\set{0, 1}^\N$ nach Satz 2.11.

  Für $a \in A_k$ bezeichnen wir mit
  \begin{align*}
    [a] \coloneqq \set{x \in X |\; \forall l \in \set{1, \dots, k}: x_i = a_i}
  \end{align*}
  die zugehörige Zylindermenge in $X$.

  Für $x \in X$ ist $V_x \coloneqq \bigcap_{k = 1}^\infty \overline{V_{x_1, \dots, x_k}}$ nichtleer und hat Durchmesser $0$, also gibt es genau ein $f(x) \in \bigcap_{k = 1}^\infty  \overline{V_{x_1, \dots, x_k}}$. Für alle $k \in \N$ bilden $(V_{a_1, \dots, a_k})_{a \in A_k}$ eine Überdeckung von $Y$, also gibt es für jedes $y \in Y$ ein $x \in X$ mit $f(x) = y$, damit ist $f$ surjektiv und wegen $d(u, v) < 2^{-k}$ folgt $f(u) \in \overline{V_{x_1, \dots, x_l}}$, woraus $d(f(u), f(v)) \leq 2^{-k}$ folgt, und $f$ ist stetig .
  Wir zeigen nun, dass $f$ die Erweiterungseigenschaft hat. Sei $\l_k$ eine Lebesguezahl von $(V_a)_{a \in A_k}$. Sei $g: X \to Y$ eine stetige Abbildung. Da $X$ kompakt und somit $g$ gleichmäßig stetig ist, gibt es eine monoton wachsende Folge $(m_k)_{k \in \N}$ so, dass $d(u, v) \leq 2^{-m_k} \; \Rightarrow \; d(g(u), g(v))< \l_k$ für $u, v \in \set{0, 1}^\N$ und $k \in \N$.

  Wenn $a \in A_{m_1}$ und $x \in [a]$, dann ist $g([a]) \subseteq B_{\l_1}(g(x))$. Also gibt es ein $\phi_1(a) \in A_1$ mit $g([a]) \subseteq V_{\phi_1(a)}$. Wir fahren rekursiv fort. Angenommen, wir haben $\phi_k: A_{m_k} \to A_k$ konstruiert, sodass $g([a]) \subseteq V_{\phi_k(a)}$. Sei
  \begin{align*}
    \psi_k:& A_{m_{k+1}} - A_{m_{k}},\\
    &(x_1, \dots, x_{m_k}, \dots, x_{m_{k+1}}) \mapsto (x_1, \dots, x_{m_k}).
  \end{align*}
  Für $a \in A_{m_{k+1}}$ gilt $[a] \subseteq [\psi_k(a)]$, also auch $g([a]) \subseteq g([\psi_k(a)]) \subseteq T_{\phi_k}(\psi_k(a))$. Dann gibt es $\phi_{k+1}(a) \in A_{k+1}$, sodass für jedes $x \in [a]$ gilt: $g([a]) \subseteq B_{\l_{k+1}}(g(x)) \subseteq V_{\phi_{k+1}(a)} \subseteq V_{\phi_k(\psi_k(a))}$. Damit gilt $(\phi_{k+1}(a))_l = (\phi_{k}(\psi_k(a)))_l $ für alle $l \in \set{1, \dots, m_k}$. Wir definieren nun $\phi:X \to X$ mit $\phi(x)_k = (\phi_k(x_1, \dots, x_{m_k}))_k$. Es gilt dann $\phi(x)_{1, \dots, k} = \phi_k(x_1, \dots, x_{m_k})$. Damit ist die Abbildung $\phi:X \to X$ stetig, denn $d(x, y)< 2^{-m_k} \; \Rightarrow \; d(\phi(x), \phi(y))<2^{-k}$. Schließlich ist $g([x_1, \dots, x_{m_k}]) \subseteq V_{\phi_k(x_1, \dots, x_{m_k})} = V_{\phi(x)_1, \dots, \phi(x)_k}$ so, dass $g(x) \in V_{\phi(x)}$ und damit $f(\phi(x)) = g(x)$.
\end{beweis}

\begin{beweis} von Satz 2.14:

  Sei $(Y, \psi)$ ein dynamisches System. Dann gibt es nach Satz 2.18 eine stetige, surjektive Abbildung $f: \set{0, 1}^\N \to Y$ mit der Erweiterungseigenschaft. Damit gibt es also ein stetiges $\phi: \set{0, 1}^\N \to \set{0, 1}^\N$ mit $\psi \circ f = f \circ \phi$.
\end{beweis}
