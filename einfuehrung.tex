
Literatur: Munkres: Topology (im Netz verfügbar)

\section{Topologische Dynamik}
\subsection{Dynamisches System}

\begin{definition}
Als (topologisches, diskretes) dynamisches System bezeichnen wir ein Paar $(X, \phi)$ bestehend aus einem kompakten metrischen Raum $(X,d)$ und einer stetigen Abbildung $\phi: X \to X.$

Wir bezeichnen mit 
\begin{align*}
\phi^n(x) = \phi \circ \dots \circ \phi(x), \quad n \in \N
\end{align*}
($n$-mal), die $n$-fache Anwendung von $\phi$ auf $x$. Insbesondere ist $\phi^0(x) = x$. Damit ist $\phi^n$ wieder eine Selbstabbildung von $X$. Ist $\phi$ invertierbar, können wir $\phi^n$ auch für negative $n$ definieren als $(\phi^{-1})^{-n}$.
\end{definition}

\begin{bemerkung} Weitere Klassen dynamischer Systeme

Im Allgemeinen erhält man Klassen diskreter dynamischer Systeme, indem man eine Klasse von Räumen mit einer gewissen Strunktur (in unserem Fall kompakte, metrische Räume) zusammen mit strukturverträglichen Selbstabbildungen dieser Räume betrachtet (in unserem Fall die Stetigkeit).
\end{bemerkung}

\begin{bemerkung} Dynamisches System als Mononidwirkung

Allgemein versteht man unter einem dynamischen System einen Raum $X$ zusammen mit einer Monoidwirkung 
\begin{align*}
  f: X \times H \to X
\end{align*}
auf $X$ (ein Monoid $H$ ist eine Halbgruppe mit einem neutralen Element). In unserem Fall wirkt der Monoid $\N_0$ (im invertierbaren Fall die Gruppe $\Z$) und $f$ ist definiert als $f(x,n)=\phi^n(x)$. Da $\N_0$ als Monoid durch das Element $1$ erzeugt wird, genügt hier eien Selbstabbildung, um die Monoidwirkung eindeutig zu bestimmen.
\end{bemerkung}

\begin{definition} Produkt, Faktor, Konjugation
  
  \renewcommand{\labelenumi}{(\alph{enumi})}
  
  \begin{enumerate}
  \item Sei $(X_i \phi_i)_{i \in I}$ eine Familie von dynamischen Systemen. Dann nennen wir das dynamische System
    \begin{align*}
      (X, \phi) \coloneqq (\prod_{i \in I} X_i, \prod_{i \in I}\phi_i)
    \end{align*}
    mit $\phi \left( (x_i)_{i \in I}\right) =  \left( \phi_i (x_i)\right)_{i \in I}$ das Produkt von $(X_i, \phi_i)_{i \in I}$
  \item Seien $(X, \phi)$ und $(Y, \psi)$ dynamische Systeme. Wenn es eine surjektive, stetige Abbildung $f:X \to Y$ gibt mit $f \circ \phi = \psi \circ f$, dann nennen wir $(Y, \psi)$ einen Faktor von $(X, \phi)$ und $f$ eine Faktorabbildung. Ist $f$ ein Homöomorphismus, dann nennen wir $f$ eine Konjugation und sagen, dass $\phi$ konjugiert zu $\psi$ ist.
    \begin{align*}
      &X \overset{\phi}{\to} X\\
      f&\downarrow \qquad \downarrow f\\
      &Y \overset{\psi}{\to}  \; Y 
    \end{align*}
  \end{enumerate}
\end{definition}

\begin{uebung}
  Sei $(X_i,\phi_i)_{i \in I}$ eine Familie von dynamischen Systemen. Dann ist $(X_i, \phi_i)$ für jedes $i \in I$ ein Faktor von $(\prod_{i\in I} X_i, \prod_{i\in I} \phi_i)$
\end{uebung}

\begin{definition} Periodizität
  
  Sei $(X, \phi)$ ein dynamisches System. Als Orbit eines Punktes $x \in X$ unter $\phi$ bezeichnen wir die Menge 
  \begin{align*}
    {\cal O}_\phi \text{}(x) \coloneqq \set{\phi^n (x) | n \in \N_0}
  \end{align*}
Ein Punkt $x \in X$ heißt periodisch bezüglich $\phi$ mit Periode $p \in \N$, wenn $\phi^p(x)=x$ gilt. Ein Punkt $x \in X$ heißt schließlich periodisch mit Vorperiode $q$ und Periode $p$, wenn $\phi^{q+p} =\phi^p(x)$.
\end{definition}
\begin{uebung}
  Zeigen Sie, dass $x$ genau dann schließlich periodisch ist, wenn ${\cal O}_\phi(x)$ endlich ist.
\end{uebung}
\begin{beispiel}Intervallabbildung
  
  Graphische Iteration [Bilder...] :
  \begin{itemize}
  \item quadratische Abbildung $\phi:[0,1], x \mapsto x^2$ auf $[0,1]$
  \item logistische Abbildung $\phi_\lambda:[0,1] \to [0,1]$, $x \mapsto \lambda x(1-x)$ auf $[0,1]$ für $\lambda \in [0,4]$.
  \item Zeltdachabbildung (chaotisch) $\phi:[0,1] \to [0,1],$
    \begin{align*}
      x \mapsto
      \begin{cases}
        2x & \text{für } x \in [0, 0.5] \\
        2-2x & \text{für } x \in (0.5, 1] 
      \end{cases}
    \end{align*}
  \end{itemize}
\end{beispiel}

\begin{uebung}
$x \mapsto 0.5(1 - \cos(\pi x))$ ist eine Konjugation zwischen der logistischen Abbildung für $\lambda = 4$ und Zeltdachabbildung.
\end{uebung}

\begin{beispiel} Rotation

  Gegeben sei eine kompakte metrische (abelsche) Gruppe $G$ (also eine abelsche Gruppe zusammen mit einer Metrik bezüglich der die Operation $+: G \times G \to G$ und $^{-1}: G \to G$ stetig sind) und ein Element $a \in G$. Dann ist die Rotation um das Element definiert als das dynamische System $(G, \phi_a)$ mit $\phi_a(x) = x + a$.
\end{beispiel}
\begin{beispiel} Winkelverdopplung auf dem Kreis

  Die Winkelverdopplung $(S^1, \phi)$ mit $\phi(x) = x^2$ ist ein dynamisches System auf dem Kreis $S^1 = \set{z \in \C | \norm{z} = 1}$.
\end{beispiel}
\begin{uebung}
  Zeigen Sie, dass das Produnkt von $n$ Rotationen auf dem Kreis eine Rotation auf dem $n$-dimensionalen Torus $\R^n/\Z^n$.
\end{uebung}
%%  Vincent hat deinen Schlüssel... {\leftarrow} Wo ist da eigentlich das Problem?  Und ich bin eine Nuss. Ich habe nämlich nicht daran gedacht, dass dein Fahhrradschlüssel an deinem Schlüsselbund dran ist, und unsere beiden Fahrräder sind grade aneinandergekettet...

\subsection{Untersystem und Minimalität}

\begin{definition}Invarianz und Untersystem

  Sei $(X, \phi)$ ein dynamisches System und $Y \subset X$. Wenn $\phi(Y) \subseteq Y$ gilt, nennen wir $Y$ \markdef{vorwärtsinvariant} unter $\phi$. Gilt $\phi^{-1}(Y) \subseteq Y$, nennen wir $Y$ \markdef{rückwärtsinvariant} unter $\phi$.

Wenn $Y$ abgeschlossen und vorwärtsinvariant ist, dann ist $(Y, \phi)$, genauer $\left( Y,\phi_{|Y}\right)$ wieder ein dynamisches System und wir sagen $(Y, \phi)$ ist ein Untersystem von $(X, \phi)$.
\end{definition}
%% Was denn?
%% Küsschen... $\varepsilon >$ = $\eps >$
%% oh kuno :D
\begin{definition} $\alpha$- und $\omega$-Grenzmenge

Sei $(X, \phi)$ ein dynamisches System. Für eine Menge $Y \subseteq X$ definieren wir die $\omega$-Grenzmenge 
\begin{align*}
  \omega(Y) \coloneqq \bigcap_{N \in \N_0} \overline{\bigcup_{n \geq N} \phi^n(Y)}
\end{align*}
(Häufungspunkte). Ist $\phi$ invertierbar, so definieren wir die $\alpha$-Grenzmenge als 
\begin{align*}
  \alpha(Y) \coloneqq \bigcap_{N \in \N_0} \overline{\bigcup_{n \geq N} \phi^{-n}(Y)}. 
\end{align*}
\end{definition}
\begin{lemma}\label{lem:spezial}Spezialfälle von $\omega$-Grenzmengen
  
Sei $(X, \phi)$ ein dynamisches System. 
\renewcommand{\labelenumi}{(\roman{enumi})}
\begin{enumerate} %\renewcommand{\labelenumi}{(\alph{enumi})}
\item Sei $Y \subseteq X$ abgeschlossen und vorwärtsinvariant unter $\phi$. Dann ist $\omega(Y) = \bigcap_{n \in \N_0} \phi^n(Y)$.
\item Wenn $Y = \set{x}$, dann ist $\omega(x) \coloneqq \omega(\set{x})$ die Menge aller Häufungspunkte von $(\phi^n(x))_{n \in \N_0}$, das heißt
  \begin{align*}
    \omega(x) = \set{z \in X | \exists (n_k)_{k\in \N} \subset \N \text{ streng monoton wachsend :} \lim_{k \to \infty} \phi^{n_k}(x) = z}
  \end{align*}
\end{enumerate}
\end{lemma}
\begin{beweis}
\renewcommand{\labelenumi}{(\roman{enumi})}
  \begin{enumerate}
  \item Da $Y$ abgeschlossen ist, ist $X$ kompakt, ist auch $\phi^m(Y)$ abgeschlossen für $m \in \N$. Da $Y$ vorwärtsinvariant ist, gilt
    \begin{align*}
      \phi^m(Y) \supseteq \phi^{m+1}(Y),
    \end{align*}
 also auch $\phi^m(Y) \subseteq \phi^n(Y)$ für $m \leq n$. Somit ist aber 
 \begin{align*}
   \overline {\bigcup_{n \geq N} \phi^n(Y)} \subseteq \overline{\phi^N (Y)} = \phi^N (Y)
 \end{align*}
und 
\begin{align*}
  \omega(Y) = \bigcap_{N \in \N_0} \phi^N(Y).
\end{align*}
\item  ($\subseteq$) Ist $z \in \omega(x)$, gilt $z \in \overline{\bigcup_{n >l} \phi^n(x)}$ für alle $l \in \N$, also existiert ein $n_l \geq l$ mit $\phi^{n_l}(x) \in B_{1/l}(z)$. Also gilt $\lim_{l \to \infty} \phi^{n_l}(x) = z$, und da $n_l \geq l$, können wir eine streng monoton wachsende Teilfolge wählen. 

($\supseteq$) Sei andererseits $(n_k)_{n \in \N}$ streng monoton wachsend mit $\phi^{n_k}(x) \to z \in X$ für $k \to \infty$. Damit gilt für jedes $K \in \N$, dass 
\begin{align*}
  z \in \overline{\bigcup_{k > K} \phi^{n_k}(x)} \subseteq \overline{\bigcup_{n \geq n_K} \phi^n (x)}.
\end{align*}
Da $(n_k)_{k \in \N}$ streng monoton wächst, folgt $z \in \omega(x)$.
 \end{enumerate}
\end{beweis}

\begin{beispiel} $\omega$-Grenzmengen von $x \mapsto x^2$ auf $[0,1]$
  Die quadratische Abbildung auf dem Einheitsintervall $([0,1], \phi)$ mit $\phi(x) = x^2$ hat für $A = [a, b] \subseteq [0,1]$ die folgenden $\omega$-Grenzmengen:
  \begin{itemize}
  \item $\omega(A)=\set{0}$ für $0\leq a < b <1$
  \item $\omega(A)=[0,1]$ für $ A = [a, 1]$ mit $0 \leq a  <1$
  \end{itemize}
\end{beispiel}

\begin{satz}\label{th:omega} Eigenschaften von $\omega$-Grenzmengen

Sei $(X, \phi)$ ein dynamisches System und $Y \subseteq X$ nichtleer. Dann ist $\omega(Y)$ nichtleer, abgeschlossen und es ist 
\begin{align*}
  \omega(Y) = \phi(\omega (Y)) = \omega(\omega(Y)) = \omega(\bar Y).
\end{align*}
\end{satz}
\begin{beweis}
  $\omega(Y)$ ist als Durchschnitt eine fallende Folge von nichtleeren, abgeschlossenen Mengen im einem kompakten Raum nichtleer und abgeschlossen. 

Es gilt (siehe Aufgabe \ref{ex:durchschnitt})
\begin{align*}
  \phi(\omega(Y))&= \phi(\bigcap_{N \in \N_0} \overline{\bigcup_{n \geq N}\phi^n(Y)})\\
&=  (\bigcap_{N \in \N_0} \overline{\bigcup_{n \geq N}\phi^{n+1}(Y)})\\
&=  (\bigcap_{N \in \N_0} \overline{\bigcup_{n \geq N+1}\phi^{n}(Y)})\\
&= \omega(Y).
\end{align*}
Damit folgt nach Lemma \ref{lem:spezial}
\begin{align*}
  \omega(Y)= \bigcap_{n \in \N_0}\phi^n(\omega(Y)) = \omega(Y)
\end{align*}
Schließlich ist offensichtlich $\omega(Y) \subseteq \omega(\bar Y)$ und mit Aufgabe \ref{ex:durchschnitt}
\begin{align*}
  \omega(\bar Y) &= \bigcap_{N \in \N_0} \overline{\bigcup_{n \geq N}\overline{\phi^n(Y)}}\\
& \subseteq \bigcap_{N \in \N_0} \overline{\bigcup_{n \geq N}\overline{\bigcup_{m \geq n}\phi^m(Y)}}\\
&= \omega(Y)
\end{align*}
\end{beweis}
\begin{uebung}\label{ex:durchschnitt}Topologische Eigenschaften kompakter Räume
\renewcommand{\labelenumi}{(\alph{enumi})}
  \begin{enumerate}
  \item Zeigen Sie die endliche Durchschnittseigenschaft.
  \item Zeigen Sie, dass für $A_0 \supseteq A_1 \supseteq A_2 \supseteq \dots$ mit $A_i \subseteq X$ kompakter metrischer Raum und $\phi: X \to X$ gilt $\phi(\bigcap_{i \in \N} A_i) = \bigcap_{i \in \N} \phi(A_i)$.
  \item $X$ kompakter, metrischer Raum $\phi:X \to X$ stetig, dann gilt für $A \subseteq X$, dass $\phi(\bar A) = \overline{\phi(A)}$.
  \item Zeigen Sie, dass jede bijektive stetige Abbildung zwischen kompakten metrischen Räumen ein Homöomorphismus (Umkehrabbildung ist auch stetig) ist.
  \end{enumerate}
\end{uebung}
\begin{uebung} Eigenschaften der $\omega$-Grenzmenge
  Geben Sie ein Beispiel für ein dynamisches System $(X, \phi)$ und $U, V \subseteq X$ sowie $U \cap V \neq \emptyset$ für das
  \renewcommand{\labelenumi}{(\alph{enumi})}
  \begin{enumerate}
  \item $\omega(U \cap V) \neq \omega(U) \cap \omega(V)$
  \item $\omega(U \cup V) \neq \omega(U) \cup \omega(V)$
  \end{enumerate}
gilt. Zeigen Sie auch, dass $\omega(X)$ gleich dem schließlichen Bild von $\phi$ ist, das heißt $\omega(x)$ ist die (bezüglich Inklusion) maximale Teilmenge $Y$ von $X$, für die $\phi(Y)= Y$ gilt.
\end{uebung}

\begin{definition}
  Ein Punkt $x\in X$ heißt \markdef{rekurrent} bezüglich einem dynamischen System$(X,\phi)$, wenn $x \in \omega(x)$.  
\end{definition}
\begin{definition}Minimalität

Ein dynamisches System $(X, \phi)$ heißt \markdef{minimal}, wenn es kein echtes, nichttriviales Untersystem hat, das heißt, wenn $\emptyset$ und $X$ die einzigen unter $\phi$ vorwärtsinvarianten Teilmengen von $X$ sind.
\end{definition}
\begin{lemma}Charakterisierung der Minimalität

Sei $(X, \phi)$ ein dynamisches System, dann sind die folgenden Aussagen äquivalent:
\renewcommand{\labelenumi}{(\roman{enumi})}
\begin{enumerate}
\item $(X, \phi)$ ist minimal.
\item Für jedes $x \in X$ gilt $\omega(x) = X$.
\item Jeder Orbit ist dicht, das heißt $\overline{{\cal O}_\phi(x)} = X$ für jedes $x \in X$.
\item Für jede nichtleere, offene Menge $U \subseteq X$ gilt $\bigcup_{n \in \N_0} \phi^{-n}(U) = X$.
\end{enumerate}
\end{lemma}
\begin{beweis}
{$(i) \Rightarrow (ii)$} 
Nach Satz \ref{th:omega} ist $\omega(x)$ nichtleer, abgeschlossen und vorwärts-invariant. Dann ist $(\omega(x), \phi)$ ein Untersystem von $(X, \phi)$, also gilt $\omega(x) = X$.

{$(ii) \Rightarrow (iii)$} Nach Lemma \ref{lem:spezial} gilt $X = \omega(x) \subseteq \overline{{\cal O}_\phi(x)} \subseteq X$. Also $\overline{{\cal O}_\phi(x)} = X$.

  {$(iii) \Rightarrow (iv)$} Sei $U$ offen, nichtleer und $x \in X$. Dann ist ${\cal O}_\phi (x)$ dicht in $X$, also gibt es $n \in \N_0$ mit $\phi^n(x) \in U$ oder anders ausgedrückt $x \in \phi^{-n}(U)$. Also ist $X = \bigcup_{n \in \N}\phi^{-n}(U)$. 

  {$(iv) \Rightarrow (i)$} Sei $V \neq X$ eine abgeschlossene, unter $\phi$ vorwärtsinvariante Teilmenge von $X$. Dann ist $U \coloneqq X \setminus V $ offen und nichtleer, es gilt also 
  \begin{align*}
    X = \bigcup_{n \in \N} \phi^{-n}(U)
  \end{align*}
Wäre $V$ nichtleer, so gäbe es $x \in V$ mit $\phi^{n}(x)\in U$ für ein $n \in \N_0$, das der Vorwärtsinvarianz von $V$ widerspricht. Damit ist $X$ minimal.
\end{beweis}

%\datum{28.Oktober 2014}
\begin{lemma} Minimalität von Faktoren
  
Jeder Faktor eines minimalen dynamischen Systems ist selbst minimal.
\end{lemma}
\begin{beweis}
  Sei $(Y, \psi)$ ein Faktor von $(X, \phi)$ mit der Faktorabbildung $f$. Wenn $W \neq Y$ eine abgeschlossene, unter $\psi$ vorwärtsinvariante Teilmenge ist, dann ist $V \coloneqq f^{-1}(W) \neq X$ abgeschlossen. Für $x \in V$ ist $f(x) \in W$ und es gilt $f \circ \phi(x) = \psi \circ f(x) \in \psi(W) \subseteq W$, also auch $\phi(x) \in f^{-1}(W) = V$. Damit ist $V$ abgeschlossen und vorwärtsinvariant unter $\phi$. Damit ist $(V, \phi)$ ein echtes Teilsystem von $(X, \phi)$, also $V \neq \emptyset$ und somit auch $W = \emptyset$.
\end{beweis}

\subsection{Mischungseigenschaften}
\begin{definition}  Nirgends dichte, magere und residuale Mengen

Sei $(X, d)$ metrischer Raum und $U \subseteq X$.  
\renewcommand{\labelenumi}{(\alph{enumi})}
\begin{enumerate}
\item $U$ heißt \markdef{nirgends dicht}, wenn das innere des Abschlusses leer ist:
  \begin{align*}
    \inner(\overline U) = \emptyset
  \end{align*}
\item $U$ heißt \markdef{mager}, wenn sie die Vereinigung nirgends dichter Mengen ist:
  \begin{align*}
    U = \bigcup_{i \in \N} U_i
  \end{align*}
und $U_i$ nirgends dicht.
\item $U$ heißt residual, wenn ihr Komplement $X\setminus U$ mager ist.
\end{enumerate}
\end{definition}

\begin{lemma}
\renewcommand{\labelenumi}{(\roman{enumi})}
  \begin{enumerate}
  \item Jede Teilmenge einer mageren Menge ist mager.
  \item Jede Obermenge einer residualen Menge ist residual.
  \item Der abzählbare Durchschnitt dichter offener Mengen ist residual.
  \end{enumerate}
\end{lemma}
\begin{beweis}
\renewcommand{\labelenumi}{(\roman{enumi})}
  \begin{enumerate}
  \item Sei $V$ mager und $U \subset V$. Dann gibt es nirgends dichte Mengen $V_i$, $ i \in \N$ mit $V = \bigcup_{i \in \N} V_i$. Dann ist 
    \begin{align*}
      U = \bigcup_{i \in \N} ( V_i \cap U)
    \end{align*}
und für $ i\in \N$ ist $V_i \cap U$ nirgends dicht, denn $\inner \overline{(V_i \cap U)} \subseteq \inner (\overline {V_i} \cap \bar U)$. Mit $\overline{A \cap B} = \bar A \cap \bar B$ und $\inner (A \cap B) = \inner A \cap \inner B$ folgt
\begin{align*}
  \inner(\bar V_i \cap \bar U) = \inner \bar V_i \cap \inner \bar U = \emptyset.
\end{align*}
Somit ist $U$ abzählbarer Durchschnitt nirgends dichter Teilmengen, also mager.
\item Betrachten der Komplemente und (i) liefert die Aussage. 
\item Wir zeigen zunächst, dass das Komplement einer offenen dichten Menge $V$ nirgends dicht ist. Wir müssen also zeigen, dass $\overline {X\setminus V} = X \setminus V$ ein leeres Inneres hat. 
Hätte $X \setminus V$ einen inneren Punkt $x$, dann gäbe es eine Umgebung $U$ von $x$, die vollständig in $X \setminus V$ liegt. Dies widerspricht daber der Dichtheit von $V$. 

Sei jetzt $V = \bigcap_{i\in \N} U_i$ der Durchschnitt offener, dichter Mengen $U_i$, $i \in \N$.
Dann ist $V^c = \bigcup_{ i \in \N} U_i^c$ abzählbare Vereinigung nirgends dichter Mengen, also mager und $V$ ist somit residual.
  \end{enumerate}
\end{beweis}
\begin{satz} Satz von Baire
  
Sei $X$ ein kompakter metrischer Raum. Dann ist jede residuale Menge dicht in $X$.
\end{satz}
\begin{beweis}
  Wir teilen den Beweis in vier Schritte.
  \begin{enumerate}
  \item Ein abzählbarer Durchschnitt offener dichter Mengen ist dicht.

    Wir zeigen, dass der Durchschnitt $U = \bigcap_{i \in \N} U_i$ offener dichter Mengen $U_i$, $ i \in \N$, dicht ist in $X$. Seien dazu $ x \in X$, $\eps > 0$ beliebig. Wir konstruieren ein $y \in B_\eps(x) \cap U$. 

Da $U_1$ dicht ist, exisitert ein $x_1 \in U_1 \cap B_\eps(x)$. Da $U_1 \cap B_\eps (x)$ offen ist, gibt es ein $\eps_1 \in (0, \frac 1 2 \eps)$ mit
\begin{align*}
  \overline{B_{\eps_1}(x)} \subseteq U_1 \cap B_\eps(x).
\end{align*}
Da $U_2$ dicht ist, existiert ein $x_2 \in U_2 \cap B_{\eps_1}(x_1) \subseteq U_1 \cap U_2 \cap B_\eps(x)$. Da $U_2 \cap B_{\eps_1}(x_1)$ offen ist, gibt es ein $ \eps_2 \in (0,\frac 1 2 \eps_1)$ mit $\overline{B_{\eps_2}(x_2)} \subseteq U_2 \cap B_{\eps_1}(x_1)$. Induktiv erhält man Folgen $(x_i)_{i\in \N} \subset X$ und $(\eps_i)_{i \in \N} \subset \R$ mit

\begin{enumerate}
\renewcommand{\labelenumi}{(\roman{enumi})}
\item  $\overline{B_{\eps_i}(x_i)} \subseteq U_i \cap B_{\eps_{i-1}} (x_{i-1}) \subset U_1 \cap \dots \cap U_i \cap B_\eps (x)$
\item $\eps_i < \frac 1 2 \eps_{i-1} < \frac 1 {2^i} \eps$
\end{enumerate}
Somit ist $(x_i)_{i \in \N}$ Cauchy-Folge und da $X$ als kompakter, metrischer Raum auch vollständig ist, gibt es ein $x_0 \in X$ mit $\lim_{i \to \infty}x_1 = x_0$. Es gilt  $x_0 \in \overline{B_{\eps_i}(x_i)} \subset U_i \cap B_\eps(x)$ für alle $i \in \N$ und deshalb $x_0 \in \left(  \bigcap_{i \in \N} U_i \right) \cap B_\eps(x)$ und somit $x_0 \in U$. 
\item  Eine abzählbare Vereinigung nirgends dichter, abgeschlossener Mengen hat keinen inneren Punkt.

Wir zeigen, dass die Vereinigung $V \coloneqq \bigcup_{i \in \N} V_i$ nirgends dichter, abgeschlossener Mengen $V_i$, $i \in \N$, keinen inneren Punkt hat. Sei dazu $W \coloneqq V^c = \bigcap_{i \in \N} W_i$ mit $W_i \coloneqq V_i^c$. Dann ist $W$ der abzählbare Durchschnitt offener dichter Mengen und mit dem ersten Schritt dicht in $X$. Angenommen, $V$ enthält einen inneren Punkt, dann gibt es eine offene Menge $U \subseteq V$. Somit folgt $U \cap V^c = U \cap W = \emptyset$ im Widerspruch zur Dichtheit von $W$.

%\datum{30. Oktober 2014}
\item Magere Mengen in $X$ haben leeres Inneres

Wir zeigen, dass eine magere Menge $V \coloneqq \bigcup_{i \in \N} V_i$, $V_i$ nirgends dicht, leeres Inneres hat. Definiere dazu $\hat V \coloneqq \bigcup_{i \in \N} \overline{V_i} $. Da $\inner (\overline{V_i}) = \emptyset = \inner (\overline {(\overline{V_i})})$, sind auch alle $\overline{V_i}$ nirgends dicht. Damit ist $\hat V$ abzählbare Vereinigung abgeschlossener nirgends dichter Mengen, hat also nach 2. leeres Inneres.
\item Jede residuale Menge in $X$ ist dicht in $X$. Sei $U$ residuale Menge, dann ist $V \coloneqq U^c$ mager, hat also nach 3. leeres Inneres, also $\inner V = \emptyset = \inner U^c$. Schließlich gilt 
  \begin{align*}
    \inner U^c = (\overline{U})^c = \emptyset,
  \end{align*}
also $\bar U = X$.
  \end{enumerate}
\end{beweis}

\begin{definition} Topologische Transitivität

Wir nennen ein dynamisches System $( X, \phi)$ topologisch transitiv, wenn es für jedes Paar von nichtleeren, offenen Teilmengen $U, V \subseteq X$ ein $n \in \N_0$ gibt, sodass $\phi^n(U) \cap V \neq \emptyset$. 
  (Mischungseigenschaft)

Wir nennen einen Punkt $x \in X$ transitiv, wenn sein Orbit unter $\phi$ dicht in $X$, also $\overline {\cO_\phi(x)}= X$. Die Menge aller transitiven Punkte von $\phi$ bezeichnen wir mit $\cT = cT(\phi)$.
\end{definition}
\begin{uebung}
  Sei $(X, \phi)$ dynamisches System. Dann ist $\phi$ transitiv.
\end{uebung}
\begin{lemma} Charakterisierung topologischer Transitivität

Sei $X$ kompakter metrischer Raum ohne isolierte Punkte. Für ein dynamisches System $(X \phi)$ sind dann äquivalent:
\begin{enumerate}
\item $(X, \phi)$ ist topologisch transitiv
\item Für nichtleere offene Mengen $U, V \subseteq X$ gibt es ein $n \in \N_0$, sodass $\phi^{-n}(V) \cap U  = \emptyset$.
\item $\cT(\phi) \neq 0$
\item $\cT(\phi)$ ist residual
\end{enumerate}
\end{lemma}
\begin{beweis}
  {(i) $\Leftrightarrow$(ii)} 
  \begin{align*}
    \phi^n(u) \cap V \neq \emptyset &\Leftrightarrow \exists x \in U : \phi^n(x) \in V\\
 &\Leftrightarrow \exists x \in U : x \in \phi^{-n}(V)\\
 &\Leftrightarrow U \cap \phi^{-n}(V) \neq \emptyset
  \end{align*}

  {(i) $\Rightarrow$(iv)} Da $X$ ein kompakter metrischer Raum ist, gibt es für jedes $ \eps > 0 $ eine endliche Menge $P_\eps \subseteq X$, sodass $X \subseteq \bigcap_{x \in P_\eps} B_\eps(x)$. Dann lässt sich jede offene Menge als Vereinigung von Mengen aus
  \begin{align*}
    \cB \coloneqq \set{B_{1/k}(x)| k\in \N, x \in P_{1/k}}
  \end{align*}
darstellen, $X$ hat also eine abzählbare Basis $\cB$.

Sei $(U_n)_{n \in \N}$ eine Abzählung der Elemente von $\cB$. Wir setzen
\begin{align*}
  V_N \coloneqq \bigcup_{i \in \N_0} \phi^{-i}(U_n) 
\end{align*}
Als Vereinigung offener Mengen ist $V_n$ wieder offen. Für jede nichtleere, offene Menge $W \subseteq X$ gibt es wegen (i) ein $i \geq 0$, so dass $\phi^{-i}(U_n) \cap W \neq \emptyset$. Damit gilt auch $V_n \cap W \neq \emptyset$. Damit ist $V_n$ offen und dicht und der Durchschnitt $V \coloneqq \bigcap_{n \in \N_0} V_n$ ist residual nach Lemma 1.25. Es bleibt zu zeigen, dass $V = \cT (\phi)$.
 In der Tat gilt: 
 \begin{align*}
   x \in V \Leftrightarrow  \forall n \in \N_0 \exists i \in \N_0: \phi^i(x) \in U_n \Leftrightarrow \forall n \in \N_0: \cO(x) \cap U_n \neq \emptyset\\
\Leftrightarrow \overline{\cO(x)} = X\\
\Leftrightarrow x \in \cT. 
 \end{align*}

(iv) $\Leftrightarrow $ (iii) Nach dem Satz von Baire ist $\cT(\phi)$ dicht, also nichtleer.

(iii) $\Leftrightarrow $ (i) Sei $x \in \cT(\phi)$ und seinen $U, V \subseteq X$ offen und nichtleer. Dann gilt es $n \in \N$ mit $\phi^m(x) \in U$. Weiterhin ist $V\setminus \set{\phi^k(x)|k \in \set{0, \dots, m}}$ nichtleer und offen, da $X$ keine isolierten Punkte hat. Damit gibt es $n> m$ mit 
\end{beweis}
%\datum{4. November 2014}
% FEHLT!
%\datum{11. November 2014}
\begin{satz}Glasnar, Weiss, Akin   

Sei $(X, \phi)$ dynamisches System transitiv und fast gleichgradig stetig. Dann gilt:
\begin{enumerate}
\item $T(\phi) = \cE \phi$
\item $(X, \phi)$ ist gleichmmäßig starr, das heißt, es gibt eine Folge $(n_k)_{k \in \N}$, sodass $\phi^{n_k}$ gleichmäßig gegen $\id_X$ konvergiert.
\item $\phi$ ist ein Homöomorphismus.
\end{enumerate}
\end{satz}
\begin{beweis}
\renewcommand{\labelenumi}{(\roman{enumi})}
  \begin{enumerate}
  \item 
  \item 
  \item Da $(X, \phi)$ transitiv ist, ist $\phi$ surjektiv nach Aufgabe1.28. Angenommen, es gibt $x, y \in X$ mit $\phi(x) = \phi(y)$ und $d(x, y) \eqqcolon \eps > 0$. Dann gibt es wegen (ii) ein $n \in \N$ mit $d(\phi^n(x), x) \leq \frac \eps 2$ und $d(\phi^n(y), y)< \frac \eps 2$, zusammen ergibt sich der Widerspruch 
    \begin{align*}
      0 = d(\phi^n(x), \phi^n(y)) \geq d(x, y)- d(x, \phi^n(x))- d(y, \phi^n(y))> \eps-\frac \eps 2 -\frac \eps 2 = 0
    \end{align*}
Also ist $\phi$ bijektiv und damit auch ein Homöomorphismus.
  \end{enumerate}
\end{beweis}
Es folgen nun drei Sätze über dynamische Eigenschaften, die Sensitivität implizieren. 
\begin{satz}
  Jedes mischende dynamische System mit mindestens zwei Punkten ist sensitiv.
\end{satz}
\begin{beweis}
  Seien $(X, \phi)$ ein mischendes dynamisches System und $u, v \in X$ zwei verschiedene Punkte in $X$. Sei 
  \begin{align*}
    \eps \coloneqq \frac {d(u,v)} 4,
  \end{align*}
dann gibt es für jedes $x \in X$ und $\delta > 0$ ein $n \in \N$ mit $\phi^n(B_\delta (x)) \cap B_\eps(u) \neq \emptyset$ und $\phi^n(B_\delta(x)) \cap B_\eps(v) \neq \emptyset$. Also gibt es $y \in B_\delta(x)$ und $z \in B_\delta(x)$ mit $d(\phi^n(y), u) <\eps$ und $d(\phi^n(z), v) <\eps$. Damit gilt schließlich 
\begin{align*}
  d(\phi^n(y), \phi^n(z))&\geq d(u,v) - d(\phi^n(y), u)- d(\phi^n(z, v)\\
&\geq 4 \eps - \eps - \eps = 2 \eps.
\end{align*}
Somit ist entweder $d(\phi^n(y), \phi^n(x))\geq \eps$ oder  $d(\phi^n(z), \phi^n(x))\geq \eps$, ist also sensititv mit Sensitivitätskonstante $\eps$.
\end{beweis}
\begin{satz}Banks et al. 1992

Sei $(X, \phi)$ ein transitives dynamisches System, für das $\per(\phi)$ dicht liegen und $X$ enthält unendlich viele Punkte. Dann ist $(X, \phi)$.
\end{satz}
\begin{beweis}
  Wir teilen den Beweis in zwei Schritte: 
  \begin{enumerate}
  \item Es gibt für alle $\eta > 0$ und alle $x \in X$ ein $q \in \per(\phi)$, sodass
    \begin{align*}
      \dist(x, \cO_\phi(q))> \eta.
    \end{align*}
Da $\per(\phi)$ dicht liegt, gibt es einen periodischen Punk $q_1$ und da $X$ unendlich viele Punkte enthält, gibt es auch einen zweiten Periodischen Punkt $q_2 \notin\cO_\phi(q_1)$ (sonst Widerspruch zu $\per(\phi)$ dicht). Definiere 
\begin{align*}
  \eta &\coloneqq \frac 1 3 \dist(\cO(q_1), \cO(q_2))\\
&= \frac 13 \min \set{d(x_1, x_2)| x_1 \in \cO_\phi(q_1), x_2 \in \cO(q_2)}>0.
\end{align*}
Sei $x \in X$ beliebig. Dann gibt es $r_1 \in \cO_\phi(q_1)$ und $r_2 \in \cO_\phi(q_2)$ mit 
\begin{align*}
  \dist(x, \cO_\phi(q_1)) = d(x, r_1),\\
  \dist(x, \cO_\phi(q_2)) = d(x, r_2).
\end{align*}
Mit der Dreiecksungleichung folgt
\begin{align*}
  \dist(x, \cO_\phi(q_1)) +   \dist(x, \cO_\phi(q_2)) &= d(x, r_1) + d(x, r_2)\\
&\geq d(r_1, r_2) \geq 2 \eta
\end{align*}
und somit
\begin{align*}
  \dist(x, \cO_\phi(q_1))> \eta
\end{align*}
oder 
\begin{align*}
  \dist(x, \cO_\phi(q_2))> \eta.
\end{align*}
\item $(X, \phi)$ ist sensitiv mit Sensitivitätskonstante $\frac \eta 4$. Wir zeigen, dass für alle $x \in X$ und alle $\delta > 0$ ein $y \in B_\delta(x)$ existiert, sodass ein $n \in \N$ existiert, sodass
  \begin{align*}
    d(\phi^n(y), \phi^n(x)) \geq \frac \eta 4. 
  \end{align*}
Da $\per(\phi)$ dicht liegt, gibt es ein $p \in \per(\phi) \cap B_{\frac \eta 4}(x)$. Wegen dem ersten Schritt gibt es ein $q \in \per(\phi)$ mit $\dist (x, \cO_\phi(q)) > \eta$. Sei $k$ die Periode von $p$. Dann ist die Menge 
\begin{align*}
  U(q) \coloneqq \bigcap_{i = 0}^k \phi^{-i}(B_{\frac \eta 4}(\phi^i(q)))
\end{align*}
eine offene Umgebung von $q$, da $\phi$ stetig ist. Da $\phi$ transitiv ist, gibt es ein $z \in B_\delta(x)$ mit 
\begin{align*}
  \phi^j(z)\in U(\phi)
\end{align*}
für ein $j \in \N$.

Definiere nun mittels der Gauß-Klammer 
\begin{align*}
  l \coloneqq \floor{ \frac j k + 1}, 
\end{align*}
das heißt $\frac j k < l \leq \frac j k +1$. Es folgt $j < lk \leq j +k $ oder äquivalent $0< lk -j\leq k$. Nach Konstruktion von $U(q)$ gilt
\begin{align*}
  \phi^{lk}(z) = \phi^{lk-j}(\phi^j(z)) \in \phi^{lk- j}(U(q))  \subseteq B_{\frac \eta 4}(\phi^{lk-j}(q)). 
\end{align*}
Mit der Beziehung $\phi^{lk}(p) = p$ ($p$ ist $k$-periodisch) und der Dreiecksungleichung folgt
\begin{align*}
  d(x, \phi^{lk- j}(q)) \leq d(x, p) + d(\phi^{lk}(p), \phi^{lk}(z)) + d(\phi^{lk}(z), \phi^{lk}(q)). 
\end{align*}
Hieraus ergibt sich, wieder mit der Dreiecksungleichung 
\begin{align*}
d(\phi^{lk}(p), \phi^{lk}(x)) + d(\phi^{lk}(x), \phi^{lk}(z)) &\geq d(\phi^{lk}(p), \phi^{lk}(z)) \\
&\geq d(x, \phi^{lk-j}(q)) - d(x, p)- d(\phi^{lk}(x), \phi^{lk-j}(q))\\
&> \eta - \frac \eta 4 - \frac \eta 4 > \frac \eta 2.      
\end{align*}
Daraus folgt
\begin{align*}
  d(\phi^{lk}(p), \phi^{lk}(x)) > \frac \eta 4
\end{align*}
oder 
\begin{align*}
  d(\phi^{lk}(x), \phi^{lk}(z)) > \frac \eta 4.
\end{align*}
Wähle $n\coloneqq lk$, sowie $y \coloneqq p$ im ersten Fall und $y\coloneqq z$ im zweiten  Fall. 
\end{enumerate}
\end{beweis}
Eine etwas stärkere Eigenschaft als Sensitivität ist die positive Expansivität.
\begin{definition}
  Ein dynamisches System $(X, \phi)$ heißt \markdef{positiv expansiv}, wenn es ein $\eps > 0$ gibt für alle $x, y \in X$ mit $x \neq y$ und es dafür ein $n \in \N_0$ gibt mit 
  \begin{align*}
    d(\phi^n(y),\phi^n(x)) \geq \eps.
  \end{align*}
$\eps$ heißt Expansivitätskonstante von $\phi$.
\end{definition}
\begin{satz}
  Sei $(X, \phi)$ ein positiv expansives dynamsiches Systema auf einem perfekten Raum $X$, has heißt, $X$ enthält keine isolierten Punkte. Dann ist $(X, \phi)$ sensitiv.
\end{satz}
\begin{beweis}
  Sei $\eps >0$ die Expansitivitätskonstante von $(X, \phi)$. Da $X$ perfekt ist, gibt es für jedes $x \in X$ und jedes $\delta > 0$ ein $y \in B_\delta(x)\setminus \set x$ und $n \in \N$ mit $d(\phi^n(x), \phi^n(y)) > \eps$.
\end{beweis}